% !TEX root =  paper.tex

\section{Related Work}
\label{sec:related-work}

Many techniques have been proposed 
for crawling dynamic web apps~\cite{Mirtaheri:2013:HistoryOfCrawlers}.
\textsc{AJAXSearch} is tailored for indexing Ajax applications
for efficient searching~\cite{Frey:2007-IndexingAjaxApps, Duda:2008:AjaxSearchTool}.
It only considers \html attributes (e.g., \texttt{onclick}) 
for detecting attached events. 
In contrast, \toolName uses a machine learning model to detect any actionable element regardless of how the events listeners are attached.
\textsc{RE-RIA}~\cite{Amalfitano:2008:ReverseEngineeringRIA} 
collects execution traces from user sessions by manual driving of the web app,
or through automated depth-first crawling in \textsc{CrawlRIA}~\cite{Amalfitano:2010:CrawlRIA}.
It constructs a model from these traces. %by clustering equivalent user interfaces based on an equivalence relation.
On the contrary, our approach does not need user sessions to drive the web app. %manual driving of the web application; it drives a general-purpose crawler to build a model of the web app.

Peng et al.~\cite{Peng:2012:GraphBased} propose to use a greedy crawling strategy.
The proposed approach assumes that 
the events that change the state of the web application by adding or removing DOM nodes are not fired,
which is far from realistic in modern web apps.
Our technique focuses on exploring the event space of the web application,
rather than providing a novel state exploration strategy.

Dincturk et al.~\cite{Dincturk:2014:ModelBasedForRIA:TWEB} propose a model-based strategy for efficient crawling,
i.e., ``to explore more states early on in the crawling'',
based on an \textit{anticipated model} of the web application.
This model allows to choose events that might lead to more efficient crawling.
The anticipated model adapts to the actual behavior of the web app during the exploration.
A model that reflects the web app's actual behavior as accurately as possible should be chosen,
e.g., a \textit{hypercube}-like model,
which assumes that the newly-explored states share the same events with the old ones.
A crawling strategy with this hypercube-based model is proposed~\cite{Benjamin:2007:StrategyHypercube:Thesis, Benjamin:2011:StrategyHypercube:ICWE};
yet the authors show that the mentioned assumption is frequently violated in real-word apps~\cite{Dincturk:2014:ModelBasedForRIA:TWEB}.
Other proposed meta-models, e.g., the \textit{Menu model}~\cite{Choudhary:2012:MCrawler:Thesis, Choudhary:2013:MenuModel:ICWE},
or the \textit{probabilistic model}~\cite{Dincturk:2012:StatisticalStrategy, Dincturk:2013:ModelBasedCrawlingStrategy} make different assumptions about the web apps under exploration.
In contrast to the model-based crawling, \toolName does not need the user to provide a model of the web application for exploration.
Plus, the implementations of the model-based crawlers require special environments to have full control on \js execution~\cite{Dincturk:2014:ModelBasedForRIA:TWEB}.
\toolName is free from such limitations and can explore any web app, in any web browser, served from any web server.

Moosavi et al.~\cite{Moosavi:2013:ComponentBasedCrawling, Moosavi:2014:ComponentBasedCrawlingICWE} propose the idea of categorizing events for effective state exploration.
However, similar to other approaches, it only considers events attached
via \html attributes.
The approach incorporates \js event handler function names to identify similar events.
\toolName predicts five types of event handlers regardless of how they are attached,
and ranks actionable elements using stylistic cues.
The proposed approach by Moosavi et al. also needs special environments (e.g., browser engines) to run. 

\crawljax~\cite{Mesbah:2012:Crawljax}
is the most cited crawler for modern web apps.
Our technique extends \crawljax's event exploration strategy using stylistic cues.

\textsc{FeedEx}~\cite{MilaniFard:2013:FeedEx}
replaces the default state exploration strategy of \crawljax. 
The strategy sorts partially-explored states using their
\js code coverage, their event path diversity, and DOM diversity.
\textsc{FeedEx} also ranks click events
based on the diversity of the explored states
observed through previous examinations of the events,
i.e., it requires multiple examinations of the same clickable.
In contrast, \toolName focuses on predicting and ranking actionables,
and not on state space exploration strategy, which is the main goal of \textsc{FeedEx}.

\textsc{Artemis}~\cite{artzi2011framework} is a general-purpose
framework for building analysis techniques for dynamic \js web apps.
In contrast to \toolName, \textsc{Artemis} 
requires a special \js engine to execute.
\textsc{JSDep}~\cite{Sung:2016:StaticDomDependencyAnalysis}
improves \textsc{Artemis}
by proposing a technique to identify DOM and event handler dependencies.
Dallmeier et al.~\cite{dallmeier2014webmate, Dallmeier:2013:WebMate} propose a black-box
approach, called \textsc{WebMate}, 
for generating tests for web apps, focusing
on testing cross-browser compatibility issues.
\textsc{WebMate} only considers \js event listeners 
from HTML attributes and for specific libraries (e.g., \textsc{jQuery}).
In contrast to \textsc{WebMate} and \textsc{Artemis}, \toolName is framework- and browser-independent
and focuses on identifying event listeners only using stylistic cues.

Thummalapenta et al.~\cite{sinha:icse2013} propose a method
for guided test generation of web apps
by exercising ``interesting''
behaviors.
An interesting behavior is defined using business rules
and business logic,
which the authors show can be more effective
in covering business rules compared to an undirected technique.
Lin et al.~\cite{Lin:2017:Similarity} propose a technique for exploring the input space
of the web app trough semantic similarity.
We believe that our technique can complement these approaches.

Borges et al.~\cite{Borges:2018:GuidingAppTestingMinedInteractionModels}
propose a technique to learn from interaction models 
to guide mobile app testing.
The proposed approach is similar to ours
in using a machine learning model
to identify interesting elements to click on; 
however, our approach uses a much richer set of stylistic and structural features, 
and also provides a mechanism for
ranking actionables 
in web apps.
Nevertheless,
our approach is also applicable to hybrid mobile apps and 
 mobile (progressive) web apps, which use \css for defining presentation semantics.
